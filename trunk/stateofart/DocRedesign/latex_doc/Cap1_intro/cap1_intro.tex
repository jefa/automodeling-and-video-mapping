\chapter{Introducción}

%No se menciona 'proyeccion sobre superficies irregulares directamente'%

La proyección de imágenes sobre una superficie plana mediante un foco luminoso es una técnica que tiene variadas aplicaciones, siendo de las más utilizadas la proyección de películas cinematográficas. En los últimos tiempos se introdujeron dispositivos de video que permiten proyectar la salida de una computadora, comúnmente utilizados en ámbitos empresariales y académicos con fines de dictado de cursos y presentaciones. En los últimos años se popularizó la proyección sobre superficies distintas a la plana.%Falta prosa%
Es en función de esta inquietud que surgió el \emph{video mapping}, técnica que se basa en la proyección de imágenes y videos sobre fachadas, edificios u otros objetos tridimensionales de modo de resaltar, ocultar o transformar regiones de interés. %Estas ultimas 2 oraciones se pueden integrar%
Esto se logra mediante la utilización de software especializado, distorsionando las imágenes y videos proyectados logrando distintos efectos visuales.

Un espectáculo de \emph{video mapping} es una expresión artística en la que un \emph{VJ}$^\dagger$ presenta su creación mediante la proyección de diversos efectos visuales que generalmente son acompañados por efectos de sonido. Para su construcción no existe un proceso definido estándar a seguir sino que cada artista utiliza sus técnicas, métodos y aplicaciones de software de terceros o propias. Aun así se identifican etapas comunes como la construcción de un modelo virtual de la escena, la producción del espectáculo y su proyección. A su vez se identifican dificultades comunes como la calibración de los proyectores al momento de la reproducción y la generación del modelo.

En este proyecto se estudian distintos métodos para facilitar la obtención del modelo virtual. En particular se relevan métodos de escaneo y reconstrucción de objetos tridimensionales y se implementa una de las estrategias estudiadas para su generación.%Ampliar y poner imagenes%
Se desarrolla también una aplicación que permite la creación, edición y reproducción de un espectáculo de \emph{video mapping} basada en un motor tridimensional$^\dagger$. Ésta combina la edición y la reproducción permitiendo visualizar en tiempo real los distintos efectos diseñados. A su vez permite distribuir la reproducción entre varias computadoras, asociadas a uno o más proyectores, y todas sincronizadas por un componente central.

%Organización del documento%
Este documento se divide en siete capítulos que se describen a continuación.

En el Capítulo 2 - Creación de un espectáculo de \emph{video mapping}, se explican las nociones básicas de la técnica y las etapas en el proceso de creación que son el modelado, la producción y la calibración. Estas etapas se presentan mediante dos enfoques: bidimensional y tridimensional.

En el Capítulo 3 - Relevamiento de aplicaciones de \emph{video mapping}, se presenta un relevamiento de las aplicaciones existentes más populares utilizadas para la creación de espectáculos.

En el Capítulo 4 - Aportes, se presentan los resultados de una serie de entrevistas realizadas a \emph{VJs} e ingenieros contactados que contribuyen a la investigación del estado del arte.

En el Capítulo 5 - \emph{VMT}, se presenta la solución propuesta, una herramienta para la edición y posterior ejecución de espectáculos de \emph{video mapping}.

En el Capítulo 6 - Obtención de geometría, se explican distintos algoritmos y técnicas para la obtención automática y la reconstrucción de geometría tridimensional.

En el Capítulo 7 - Conclusiones y trabajo futuro, se resumen las conclusiones y posibles líneas de trabajo futuro.
