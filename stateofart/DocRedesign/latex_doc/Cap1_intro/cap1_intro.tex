\chapter{Introducción}

La proyección de imágenes amplificadas sobre una superficie plana mediante un foco luminoso es una técnica que tiene variadas aplicaciones, siendo de las más utilizadas y antiguas la proyección de películas cinematográficas. En los últimos tiempos se introdujeron dispositivos de video que permiten proyectar la salida digital de una computadora, comúnmente utilizados en ámbitos empresariales y académicos con fines de dictado de cursos y presentaciones. No fue hasta estos últimos años que se comenzó a pensar en ir mas allá de la superficie plana de proyección. Es en función de esta inquietud que surgió el \emph{video mapping*}, técnica que utiliza la proyección de imágenes y video sobre fachadas, edificios
u otros objetos tridimensionales de modo de resaltar, ocultar o transformar regiones de interés. Se realizan efectos mediante la distorsión de las imágenes y videos proyectados utilizando software especializado.

Un espectáculo de \emph{video mapping} es una expresión artística en la que un \emph{VJ}\footnote{Ver glosario.} presenta su creación mediante la proyección de diversos efectos visuales que generalmente son acompañados por efectos de sonido. Para su construcción no existe un proceso definido a seguir sino que cada artista utiliza sus técnicas, metodologías o aplicaciones de software de terceros o propias. Aun así se identifican etapas comunes que son la construcción de un modelo virtual de la escena*, la producción del espectáculo y su proyección. De este mismo modo se identifican dificultades comunes como la calibración* de los proyectores al momento de la reproducción* y el modelado realizado de forma manual.

En este proyecto se estudian distintos métodos para facilitar la obtención del modelo virtual. En particular se relevan métodos de escaneo* y reconstrucción de objetos tridimensionales y se implementa una de las estrategias estudiadas para la generación de objetos tridimensionales virtuales. %reconstrucción?
Se desarrolla también una aplicación que permite la creación, edición y reproducción de un espectáculo de \emph{video mapping} basada en un motor tridimensional*.
Esta combina la edición y la reproducción permitiendo visualizar en tiempo real los distintos efectos diseñados. A su vez permite distribuir la reproducción
entre varias computadoras, asociadas estas a uno o mas proyectores, y todas sincronizadas por un componente central.