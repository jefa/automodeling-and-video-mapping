\chapter{Estado del arte}

\section{Creación de un espectáculo de \emph{video mapping}}

En la creación de un espectáculo de \emph{video mapping} se identifican tres etapas que son el modelado de la escena, la producción del espectáculo y la proyección del mismo.

En el modelado de la escena se obtiene una representación virtual y abstracta de los objetos reales sobre los que se proyectará. Este modelo será la base para trabajar en posteriores etapas y sobre el cuál se diseñará el espectáculo.

En la siguiente etapa se realiza la producción del espectáculo que consiste en aplicar distintos efectos visuales sobre los objetos modelados y orquestar la ejecución de los mismos. En esta etapa también se diseña y produce la musicalización que será utilizada durante todo el espectáculo.
%Orquestar no aplica para toques en vivo, hay que explicar este caso también y reescribir.%

Es en la proyección del espectáculo en donde se puede contemplar el resultado de los distintos efectos visuales proyectados sobre las superficies acompañados por efectos de sonido.
Para lograr la correspondencia en la proyección de los objetos del modelo con las superficies se debe calibrar la proyección. Esta correspondencia se logra modificando el modelo de la escena, ajustando la posición y orientación de los proyectores, y ajustando los parámetros intrínsecos$^\dagger$ de la proyección.
%Esta etapa no es necesariamente una proyeccion final sino que se puede simular%

Cada una de estas etapas se puede abordar con un enfoque bidimensional o tridimensional.
Con un enfoque bidimensional el modelo de la escena es una proyección en perspectiva de los objetos a proyectar, por ejemplo, una fotografía. Con un enfoque tridimensional el modelo se mantiene independiente de un punto de vista, representándose con un conjunto de objetos virtuales tridimensionales.

%explicacion de pie de diagrama:redactarlo mejor, El modelo generado en 2d depende del punto de vista del proyector y puede no parecerse a la realidad. El modelo generado 3d se asimila a la realidad y luego con la cámara virtual ajusto el punto de vista del proyector.

\begin{figure}[H]
  \centering
    \includegraphics[width=0.7\textwidth]{./Cap2_videomapping/proy2dvs3d}
  \caption{Esquema de proyección.}
  \label{fig:proy2dvs3d}
\end{figure}

\subsection{Enfoque bidimensional}
\subsubsection{Modelo}
Un modelo bidimensional refleja lo que vería un observador desde un punto de vista fijo.
%si se quieren agregar hay en el SVN imagenes de proyeccion perspectiva y paralela  VER
Técnicamente es el resultado de una proyección en perspectiva \cite{LibroCompGrafica} sobre un plano de vista de los elementos de la superficie a modelar.

Este punto de vista debe ser considerado al posicionar y orientar el proyector que reproducirá el espectáculo.
La posición, orientación y campo de vista del proyector definirán además la sección de superficie sobre la que se proyectará.
En caso de utilizar más de un proyector cada uno de estos será posicionado en un lugar diferente y por lo tanto será necesario un modelo por cada uno de ellos.

\begin{figure}[H]
  \centering
    \includegraphics[width=0.7\textwidth]{./Cap2_videomapping/diagrama-2proyectores}
  \caption{Proyectores y sus puntos de vista.}%referenciar imagen%
  \label{fig:diagrama-2proyectores}
\end{figure}

Son ejemplos de modelos una fotografía, un plano arquitectónico de una fachada o figuras geométricas bidimensionales que representan las secciones de las superficies. En algunos casos estos se combinan para obtener como resultado un único modelo.
\begin{itemize}
  \item Una fotografía. Ubicando la cámara de forma que su punto de vista y el del proyector coincidan minimizará los ajustes necesarios en la etapa de calibración.%extender y que no quede tan en el aire lo de ajustes de calibracion%
  \item Un plano arquitectónico contiene información exacta de las medidas de la superficie que representa en una escala dada. Generalmente utiliza el método de proyecciones paralelas \cite{LibroCompGrafica} sobre un plano de proyección. Para utilizar el plano arquitectónico como modelo se debe transformar de forma que coincida con la proyección en perspectiva desde el punto vista que estará ubicado el proyector.
  \item Las figuras geométricas modelan sectores de la superficie donde se proyectará. Un método para generar las figuras geométricas consiste en utilizar un proyector y herramientas de software que permiten delinear el contorno de las secciones de la superficie en tiempo real. Se observa el resultado de cada figura generada proyectada sobre la superficie ajustando el modelo en el momento de la construcción.%%%figuras geometricas queda muy abstracto, hay que llevarlo mas al tema de representacion digital de quads%%%
\begin{figure}[H]
  \centering
    \includegraphics[width=0.7\textwidth]{./Cap2_videomapping/RepresentacionconfigurasGeometricas}
  \caption{Representación con figuras geométricas.}
  \label{fig:RepresentacionconfigurasGeometricas}
\end{figure}

Al usar el proyector para obtener el modelo queda incorporada la perspectiva del mismo y si no se modifica su posición y orientación no sería necesaria otra calibración.
Otra opción es dibujar las figuras con una fotografía o plano de fondo. En este caso la construcción de las figuras geométricas se realiza delineando el contorno de la superficie en la fotografía o plano.
Una forma automática de generar el modelo es utilizando técnicas de visión por computadora$^\dagger$, por ejemplo, en base a algoritmos de reconocimiento de aristas \cite{ArticuloAutom2dmodel}.
\end{itemize}

\subsubsection{Producción del espectáculo}
La producción del espectáculo en dos dimensiones consiste en definir efectos visuales sobre regiones de un espacio bidimensional discreto$^\dagger$ representadas en el modelo de la escena. Este espacio bidimensional discreto se representa con coordenadas de pantalla que identifican cada uno de los píxeles$^\dagger$ del área de trabajo. Los efectos visuales se logran realizando cualquier animación computacional que genere una salida gráfica como pueden ser videos e imágenes.
En esta etapa, además de definir los efectos, se planifica en qué momento se mostrarán cada uno de ellos, pudiendo sincronizarse con la música que forma parte del espectáculo.
%Mencionar caso de espectaculos en vivo%

En computación gráfica se utilizan texturas para proyectar videos e imágenes sobre regiones del área de trabajo. Las texturas son mapas de bits$^\dagger$ utilizados para cubrir la superficie de un objeto virtual. Estos mapas de bits pueden ser generados a partir de imágenes, videos, o incluso dinámicamente mediante algoritmos permitiendo así crear efectos visuales como, por ejemplo, la transición de un color a otro.
Cuando se utilizan videos estos pueden ser generados teniendo en cuenta la superficie donde se está proyectando y el punto de vista desde donde se contemplará el espectáculo. Esto es particularmente importante cuando el contenido a mostrar pretende crear una ilusión tridimensional, pues la perspectiva debe coincidir con la de los espectadores.%ampliar o explicar mejor%
\begin{figure}[H]
  \centering
    \includegraphics[width=0.7\textwidth]{./Cap2_videomapping/3dillusion}
  \caption{Izq. Ilusión 3D lograda. Der. No se logra la ilusión.}%ref%
  \label{fig:3dillusion}
\end{figure}

En este contexto se habla de mapeo, no como la salida a través de un equipo proyector, sino como la operación que logra una correspondencia entre una textura y una figura geométrica que no necesariamente coinciden en tamaño y forma. Para esto se definen coordenadas de textura en cada vértice de la figura geométrica que referencian distintas ubicaciones dentro de la misma.
Las coordenadas de las texturas tienen dos componentes: una horizontal y una vertical llamadas U y V. Si el valor de estas componentes se normaliza entre 0 y 1 entonces la esquina superior izquierda de la textura se corresponderá con la coordenada (0,0), la superior derecha con (1,0), la inferior izquierda con (0,1) y la inferior derecha con (1,1).
%agregar referencia a correspondencia de textura en un objeto, ver para imagen http://en.wikipedia.org/wiki/UV_mapping
Los vértices de una figura geométrica se asocian con coordenadas UV que definen el punto de la textura que se corresponde sobre el vértice. Mediante interpolación se logra mapear toda la textura a la figura geométrica.
Si bien es posible mapear una textura a cualquier figura geométrica, esta correspondencia es más directa utilizando un cuadrilátero ya que a cada uno de los vértices se lo hace corresponder con una esquina de la textura. A su vez el cuadrilátero es la figura básica en las aplicaciones\footnote{Las aplicaciones relevadas utilizan el cuadrilátero como figura básica. Ver apéndice: Aplicaciones relevadas.} de \emph{video mapping}.
Estos cuadriláteros se utilizan como piezas constructoras del espectáculo, cubriendo sectores del modelo sobre los cuales luego se aplican las texturas permitiendo crear los distintos efectos visuales.

\subsubsection{Calibración}
La calibración se utiliza para ajustar el modelo con la superficie que representa. Inicialmente se fija la posición y orientación del proyector y luego, con ayuda de herramientas de software, se aplica la transformación geométrica homografía\footnote{Ver sección: Obtención de geometría} a la proyección resultante, para lograr la correspondencia.
En caso de haber modificaciones en la posición y orientación del proyector la calibración deberá realizase nuevamente. %reescribir esta oración%

Los ajustes necesarios varían dependiendo del método utilizado para obtener el modelo. En caso de utilizar una fotografía, el proyector deberá ser ubicado de forma tal que el punto de vista de la cámara con la que se tomó coincida con el del proyector y así lograr la coincidencia del centro de proyección. Igualmente son necesarios ajustes ya que los lentes de la cámara y el proyector no necesariamente coinciden en el ángulo de visión \cite{LibroCompGrafica2}\cite{LibroPhotographicOptics}. Con el método de generación de figuras geométricas, en el que se modelan las secciones de la superficie a mapear utilizando el mismo proyector, los ajustes se reducen a lograr la misma posición y orientación que tenía el proyector al momento de la captura de secciones, ya que las deformaciones relacionadas a los parámetros intrínsecos fueron implícitamente consideradas durante el proceso de captura.

\subsection{Enfoque tridimensional}
\subsubsection{Modelo}
Para el modelado tridimensional se pueden utilizar representaciones de cuerpos y superficies tridimensionales representadas por mallas$^\dagger$ de polígonos \cite{Mesh_building}, comúnmente utilizadas en disciplinas como cartografía, visión computacional y computación gráfica. A diferencia de un modelo bidimensional, éste no depende de un punto de vista, lo que permite al diseñador visualizar la escena desde diferentes ángulos. Las entidades que conforman la malla son vértices, aristas, caras y atributos numéricos que representan la posición y normales de los vértices, coordenadas de textura y colores. La topología de la malla puede variar, por ejemplo, los polígonos que la componen pueden tener distinta cantidad de vértices.

\begin{minipage}{0.35\textwidth}
\begin{flushleft} \large
\begin{figure}[H]
  \centering
    \includegraphics[width=0.7\textwidth]{./Cap2_videomapping/EjemploMallaTriangular}
  \caption{Mallas triangulares.}%ref%
  \label{fig:mallas1}
\end{figure}
\end{flushleft}
\end{minipage}
\begin{minipage}{0.45\textwidth}
\begin{flushright} \large
\begin{figure}[H]
  \centering
    \includegraphics[width=0.75\textwidth]{./Cap2_videomapping/EjemploMalla4Vertices}
  \caption{Mallas de cuadriláteros.}%ref%
  \label{fig:mallas2}
\end{figure}

\end{flushright}
\end{minipage}

Es muy común que las mallas contengan información redundante, para lo que existen técnicas como la de \emph{remeshing} que mediante la utilización de algoritmos específicos reducen la cantidad de vértices de la malla sin perder la representatividad de la superficie.
%si se quiere se pone esta referencia:Survey of Polygonal Surface Simplification Algorithms Paul S. Heckbert and Michael Garland doc/papers/Mesh_building/Survey of Polygonal Surface Simplification Algorithms.pdf

Los modelos tridimensionales se generan utilizando distintas técnicas:
\begin{itemize}
  \item Modelado utilizando polígonos: a partir de mallas que representan figuras primitivas se podrán construir nuevas mallas, por ejemplo aplicando operaciones de unión o resta.
  %Que es union y resta?, explicar esto o algo mas general, union y resta son solo dos ejemplos y no los mas representativos%
  También se aplican transformaciones que modifican las aristas, vértices o caras, aproximando el resultado a la superficie que se desea modelar.
  \item  Modelado utilizando curvas: a partir de una jaula creada por curvas se aplican transformaciones para modificarla, manipulando sus puntos de control. Es utilizado en modelado de automóviles, edificios y mobiliario, entre otros.
  %programas: Maya, 3D Studio Max
  \item Esculpido digital: técnica digital que simula el esculpido convencional, en la que software especializado provee una interfaz para modificar el modelo de forma detallada, oprimiendo y resaltando zonas de la superficie. Es utilizado para lograr efectos especiales en video juegos y películas logrando figuras y texturas complejas, entre otras.
  %programas: 3D-Coat, Zbrush, y Mudbox
  \item Reconstrucción a partir de fotografías: se obtiene la representación de la superficie mediante mediciones de los objetos fotografiados. Conociendo la escala de la imagen se extrapola y se obtiene la distancia entre dos puntos en la superficie. Es usada en arquitectura, ingeniería, geología, arqueología, etc.
  %esta técnica es llamada fotogrametría en dos dimensiones, estereofotogrametría para obtener información tridimensional
  \item Reconstrucción utilizando hardware especializado: utilizando escáneres tridimensionales se obtiene una nube de puntos$^\dagger$ que representa la superficie. Generalmente la cantidad de información obtenida es densa provocando redundancia. Es por esto que se utilizan algoritmos especializados para reducir la nube de puntos.
  %explicar que el ruido tambien genera informacion que se debe depurar%
  %se puede poner una nota al pie, que esta técnica se ampliará en el capítulo Reconstrucción

\end{itemize}
\subsubsection{Producción del espectáculo}
La producción del espectáculo en tres dimensiones agrega un nivel de abstracción adicional a la producción bidimensional. Esto permite que el diseñador cree un espectáculo transformando directamente los objetos del modelo tridimensional y no sus perspectivas como en el anterior enfoque.
Esto plantea un cambio en la forma de trabajar en el espectáculo y de planificar la producción, ya que el diseñador no estará restringido a considerar ubicación alguna de los proyectores. Esta preocupación se traslada a etapas posteriores.

El modo de trabajo se basa en mapear texturas sobre las caras de los objetos tridimensionales de forma análoga a como se realiza sobre figuras bidimensionales. En este tipo de modelos podría ser deseable mapear una textura de forma que abarque varias caras del mismo objeto tridimensional, por ejemplo, la superficie de un cilindro. Para lograrlo se deben mapear distintos sectores de la textura en cada una de las caras que conforman la superficie, utilizando coordenadas de textura en cada uno de los vértices. La diferencia está en que en este enfoque los vértices no se encuentran necesariamente en el mismo plano, por lo que ajustar una textura bidimensional a este tipo de superficies no es tan directo. Para ello existen técnicas que asisten en la tarea de definir las coordenadas de textura. Una de ellas consiste en aproximar la superficie por una primitiva conocida más simple como puede ser un cubo, cilindro, esfera o plano. Para estas superficies existen funciones matemáticas que proyectan cada punto de la superficie a un plano. De esta forma se pueden determinar las coordenadas $UV$. Un ejemplo de esto para esferas son las proyecciones utilizadas para representar el planisferio como la proyección cilíndrica equidistante \cite{flatteningTheEarth}. Otra técnica es $UV$ \emph{unwrapping} que consiste en desenvolver los vértices de la superficie, aplanándolos sobre la textura. De esta forma el mapeo se realiza de forma más intuitiva ya que la superficie aplanada es bidimensional al igual que la textura. Esta técnica es soportada por la mayoría de los programas de edición de gráficos tridimensionales que proveen distintas herramientas para ajustar la forma en que se desenvuelve la textura.

En la escena tridimensional los objetos son visualizados utilizando cámaras virtuales que fijan un punto de vista. En la salida gráfica se representará entonces la escena desde la perspectiva de una cámara virtual, permitiendo visualizarla desde distintos ángulos. Esto posibilita definir el punto de vista del proyector que reproducirá la salida gráfica, y no se restringe a realizarlo antes de producir el espectáculo sino que una vez producido es posible observar el resultado y elegir el punto de vista que más agrade.
Producir el espectáculo transformando directamente los objetos y no sus perspectivas permite escalar con mayor facilidad en la cantidad de proyectores. Lo único que habría que agregar serían cámaras virtuales que definan más puntos de vista sin modificar el modelo ni los efectos producidos.

Si bien se está abordando un enfoque tridimensional de la producción del espectáculo, existen casos en que es más sencilla la implementación de ciertos efectos tomando un enfoque en dos dimensiones, sobre todo cuando estos se aplican sobre regiones planas de la superficie. Es por esto que es muy común combinar ambos enfoques y utilizarlos de acuerdo a las necesidades del diseñador. Cabe aclarar que al combinar los enfoques se pierden las ventajas de definir el punto de vista del proyector luego de la producción por lo que los objetos bidimensionales y efectos sobre estos se deben definir una vez fijada la ubicación de los proyectores.

\subsubsection{Calibración}
En este proyecto se desarrolla un método de calibración cuyo objetivo es obtener la posición del proyector con respecto a un sistema de coordenadas ubicado en un punto elegido relativo a la escena. Para ello se utiliza una superficie de calibración que puede existir ya en la escena o puede ser ubicada sobre ésta de forma temporal. Esta superficie de calibración deberá ser un plano rectangular en donde uno de los vértices será el centro de coordenadas que se desea determinar. La orientación del plano determinará la alineación del centro de coordenadas con sus ejes $X$ e $Y$ alineados con dos de los bordes del plano y el eje $Z$ perpendicular a éste. Se utiliza además un proyector asumiendo el modelo \emph{pinhole}\footnote{Ver sección con modelo \emph{pinhole}}.

Para encontrar el sistema de coordenadas buscado, se resuelve el problema opuesto que es encontrar la posición del punto que será origen del nuevo centro de coordenadas con respecto al centro de proyección ubicado dentro del proyector. Este método utiliza tres de los vértices de la superficie de calibración que formarán una base del nuevo sistema de coordenadas. Las medidas de la superficie de calibración son conocidas por lo que la distancia entre sus vértices también lo son. Desde el centro del proyector se proyectan rayos de luz, tres de los cuales pasan por los vértices de la superficie de calibración y también por el centro de proyección con coordenadas $(0, 0, 0)$. Para obtener la ecuación de estos rayos solo se precisa un punto distinto al origen. La ecuación de este punto puede ser obtenida utilizando las coordenadas en pantalla del píxel que se proyecta en cada vértice de la superficie de calibración.
%Traducir la imagen%
\begin{figure}[H]
  \centering
    \includegraphics[width=0.7\textwidth]{./Cap2_videomapping/CalibrationSketch}
  \caption{Esquema de calibración.}
  \label{fig:CalibrationSketch}
\end{figure}
Usando las coordenadas en la pantalla de estos puntos y otros parámetros intrínsecos del proyector que son su resolución y el ángulo de proyección, es posible encontrar las coordenadas del punto con respecto al centro de proyección y por tanto la ecuación del rayo.
\[
a(t) = \vec{P} . t,	\mbox{ ecuación del rayo en dónde } \vec{P} \mbox{ es cualquier punto del rayo.}
\]
\[
\begin{cases}
P(X) = \frac{res_x}{2} - s_x \\
P(Y) = \frac{res_y}{2} - s_y \\
P(Z) = \frac{res_y}{2 \cdot \tan \frac{fov_y}{2}}
\end{cases}
\]
en donde $s_x$ y $s_y$ son las coordenadas del \emph{pixel}, $res_x$ y $res_y$ son la resolución horizontal y vertical del proyector y $fov_y$ es el ángulo de proyección vertical del proyector.

Aplicando esta ecuación a cada uno de los puntos a determinar se obtienen las ecuaciones paramétricas de los tres rayos. Vale aclarar que estos puntos $P$ son puntos de los rayos pero no necesariamente coinciden con los vértices de la superficie de calibración. Queda hallar las coordenadas de los vértices determinando el parámetro $t$ para cada ecuación. Para hallar este parámetro se resuelve un sistema de ecuaciones utilizando las tres ecuaciones de los rayos y las distancias conocidas entre estos puntos.
% Sistema de ecuaciones %
\[
\begin{cases}
\lVert{a(t_0) - b(t_1)} = j\rVert \\
\lVert{b(t_1) - c(t_2)} = k\rVert \\
\lVert{c(t_2) - a(t_0)} = l\rVert
\end{cases}
\]
siendo $a$, $b$ y $c$ los rayos y $j$, $k$ y $l$ las distancias entre las parejas de puntos.
La solución será entonces los parámetros $t_0$, $t_1$ y $t_2$ que satisfacen el sistema de ecuaciones no lineal.
Una aproximación a esta solución se puede obtener utilizando métodos numéricos, por ejemplo, el método \emph{trust-region-dogleg}\cite{TrustRegionDogleg}.
Una vez hallados los valores para $t_0$, $t_1$ y $t_2$, las coordenadas de los vértices de la superficie de calibración son $a(t_0)$, $b(t_1)$ y $c(t_2)$.
La base para el sistema de coordenadas con origen en el punto $P$ es:
\[
x' = \frac{b(t_1) - a(t_0)}{\lVert b(t_1) - a(t_0) \rVert},\quad y' = \frac{c(t_2) - a(t_0)}{\lVert c(t_2) - a(t_0)\rVert},\quad z' = -x' \times y'
\]
La ubicación del proyector en este nuevo sistema se obtiene proyectando cualquiera de los rayos en la nueva base de coordenadas.

\section{Obtención de geometría}

En este capítulo se presenta la reconstrucción tridimensional automática de una superficie utilizando técnicas computacionales.
Se introducen distintos métodos que permiten la construcción automática de modelos, discutiendo sus características según propiedades de la superficie a representar y tecnologías utilizadas.
La construcción del modelo se presenta en dos etapas. Inicialmente se obtiene una nube de puntos correspondiente a la superficie utilizando técnicas y dispositivos para este propósito para luego procesar la información obtenida y construir una malla tridimensional. Mediante este procesamiento se obtienen propiedades adicionales como grupos de puntos que representan caras de una malla y las normales que identifican la orientación de la superficie.

\subsection{Obtención de la nube de puntos}

En esta sección se estudian los fundamentos matemáticos, técnicas y dispositivos relacionados al problema de la correspondencia de puntos de la realidad con puntos bidimensionales en el plano imagen, obtenido por una cámara, y la posterior determinación de su profundidad. 

\subsubsection{Correspondencia}

\subsubsection{Visión estéreo}

\cite {StereoReview} El análisis de imágenes de video en estéreo se ha utilizado como método para reconstruir la estructura tridimensional de una escena, este método no utiliza luz auxilar, por esta razón es un método pasivo. El análisis de las imágenes se realiza por medio de algoritmos que han evolucionado en los últimos tiempos logrando la reconstrucción de escenas rápidamente, es por esto que esta técnica es muy utilizada en implementaciones que necesitan respuesta en tiempo real.
En la reconstrucción utilizando el modelo de cámara estéreo se identifican dos pasos a seguir, primero resolver el problema de la correspondencia, esto es dadas dos imágenes de la escena para todos los puntos de una imagen se obtiene el punto correspondiente en la otra imagen y se calcula la desigualdad de cada punto, es decir la distancia en \emph{pixels} entre los puntos que se corresponden.
Y segundo resolver el problema de triangulación\footnote{ver anexo de método de triangulación}, teniendo en cuenta la desigualdad definida en la correspondencia utilizando el método de triangulación se obtiene la distancia focal de cada punto a cada una de las cámaras, de esta forma se calcula la profundidad del punto.
\begin{figure}[H]
  \centering
    \includegraphics[width=0.5\textwidth]{./Cap2_videomapping/stereo.PNG}
  \caption{geometía estéreo con ejes paralelos}%Structure from Stereo- A Review  Umesh R. Dhond and J.K.Aggarwal 1989%
  \label{fig:Stereo}
\end{figure}
Un caso simple sería con los ejes ópticos de las cámaras paralelos y los planos de imagen coplanares, como se muestra en la figura, en este caso las líneas epipolares\footnote{ver anexo de geometría epipolar} correspden a las filas en el \emph{frame buffer} y la correspondencia de los puntos es buscada en esas filas.
\begin{figure}[H]
  \centering
    \includegraphics[width=0.5\textwidth]{./Cap2_videomapping/epipolar3.PNG}
  \caption{dos imágenes con las líneas epipolares y puntos que se corresponden}%Geometía de Cámaras  StereoReview pag 241. fig 9.3%
  \label{fig:Stereo2}
\end{figure}

En los casos más comunes no se tienen estas características, y por ello es necesario introducir parámetros en los cálculos que hacen el ajuste del modelo ideal, modelo \emph{pinhole} al modelo de las cámaras reales. 

Existen variedad de algorimos que solucionan el problema de correspondecia y cálculo de desigualdad entre los puntos de las dos imágenes se clasifican según diferencias en las imágenes destacándose dos estrategias:
\begin{itemize}
   \item \emph{separate area}: basada en correlación del brillo e intensidad. Se utilizan patrones de brillo aplicados a un píxel y sus vecinos utilizando principio de localidad. Las diferencias en la perspectiva de la imagen o cambios en luminosidad absoluta de la escena pueden generar errores.
   \item \emph{features}: las características usadas para la correspondencia son aristas, puntos o segmentos dadas por cambios de intensidad de la imagen. Esta estrategia es más estable ante la variación de luminosidad absoluta y en la práctica la correspondencia es más rápida.
\end{itemize}
La principal desventaja de este método es que en caso de oclusión, hay regiones que no tienen correspondencia en las dos imágenes por esto no se puede resolver el problema de correspondencia para estas regiones.

\subsubsection{Luz estructurada}

El método basado en la técnica de luz estructurada consiste en un sistema formado por una o mas camaras de video mas un proyector que despliega distintos patrones sobre la superficie a modelar. Los patrones son capturadas por la o las cámara de video para su posterior analisis. Mediante patrones codificados apropiadamente para la escena siendo observada, se resuelve el problema de correspondencia de los puntos de la escena con los capturados, y luego, mediante un analisis de las deformaciones de los patrones se obtiene información tridimensional de la posición, orientación y textura de la superficie\cite{SLightPatterns}.

\paragraph{Clasificación de patrones}

Los patrones proyectados en esta técnica son variados y se clasifican en cuatro tipos: punto (\emph{singled scanned dot}), línea (\emph{slit line}), grilla (\emph{grid}) y matriz de puntos (\emph{dot matrix}).
Dependiendo del tipo de objeto y superficie a escanear pueden ocurrir problemas de oclusión, baja reflexión y puntos reflejados fuera del alcance de la cámara. Como consecuencia, existen pérdidas de puntos proyectados que no tienen proyección en el plano imagen.
Estos problemas se pueden solucionar utilizando patrones codificados adecuados. Se distinguen tres grupos de patrones clasificados: dependencia temporal, propiedades de la luz proyectada y discontinuidad de profundidad de la superficie proyectada\cite{SLightCorrespondence}.
\begin{itemize}
   \item Dependencia  temporal:
   \begin{itemize}
	\item Estática: el patrón es limitado para escenas estáticas y son necesarias proyecciones de varios patrones distintos. El movimiento de cualquier objeto de la escena mientras se realiza la obtención de los patrones proyectados producirá un error de correspondencia.
	\item Dinámica: los objetos en la escena se pueden mover y se utiliza un único patrón de proyección.
   \end{itemize}
   \item Propiedades de la luz proyectada:
   \begin{itemize}
	\item Binaria: cada uno de los puntos del patrón tiene dos posibles valores codificados con 0 y 1 respectivamente. Este valor representa opacidad y transparencia, ausencia o presencia de la luz proyectada en el objeto.
	\item Escala de grises: cada punto del patrón tiene asociado un valor de gris que representa el nivel de trasparencia o nivel de opacidad del punto para la luz proyectada. Son necesarios dos pasos, primero se obtiene una imagen de la escena iluminada con la misma luz, o sea, sin variar la intensidad. Luego, se obtiene la referencia de luz necesaria para cancelar el efecto de reflejo de la superficie que dependerá del tipo de superficie. La necesidad de estos dos pasos contribuye a que este patrón también sea clasificado como estático.
	\item Color: cada punto del patrón es asociado con un valor de tono debiendo estos ser bien diferenciados para alcanzar una segmentación eficiente. Este tipo de patrones son limitados por el color de la escena. Si presenta objetos de colores altamente saturados se producen pérdidas de regiones en el paso de segmentación que luego provoca errores en la decodificación.
   \end{itemize}
   \item Discontinuidad en profundidad de la superficie proyectada:
   \begin{itemize}
	\item Periódica: la codificación se repite periódicamente a lo largo del patrón. Esta técnica se utiliza para reducir el número de bits que codifican el patrón. Como limitante, la profundidad del objeto no puede ser mayor que la mitad de la longitud del período.%?%
	\item Absoluta: cada columna o fila del patrón proyectado tiene una única codificación. No sufre dependencia de discontinuidad de profundidad.
   \end{itemize}
\end{itemize}

\paragraph{Calibración}

En esta sección se describe un método originalmente propuesto por Zhang [Zha00] para la calibración de una cámara y un proyector con el objetivo de capturar correctamente los patrones proyectados para su posterior análisis. 

La calibración de la cámara requiere inicialmente estimar los parametros del modelo de cámara pinhole. Por lo tanto se deberán obtener los parámetros intrínsicos como distancia focal, punto principal y factores de escala, y los extrínsicos definidos por la matriz de rotación y vector de traslación de un punto en el espacio al sistema de coordenadas de la cámara. 
%We recommend the reader review [HZ04, MSKS05] for an in-depth description of camera models and calibration methods
%el modelo pinhole que nosotros presentamos no incluye los extrinsicos (matrices de traslacion y rotacion) 

Básicamente la calibracion de la cámara requiere la captura de una secuencia de imagenes de un objeto simple, con un conjunto de caracteristicas fijo y distinguibles, y con un desplazamiento tridimensional conocido. Esto permite que cada imagen capturada durante el proceso de calibracion provea de un conjunto de correspondencias de los puntos tridimensionales de la escena a puntos bidimensionales en el sistema de coordenadas de la camara. Particularmente, en el método de Zhang, el objeto conocido que se observa es un tablero de damas plano en una o mas orientaciones. De la secuencia capturada se pueden obtener los parametros intrinsicos y luego, utilizando una sola toma de la secuencia,  se obtienen los parametros extrinsicos.

Para la calibracion del proyector, este se modela como el inverso de una cámara, teniendo en cuenta que la luz viaja en la dirección opuesta y que un punto en el plano de la imagen se mapea a un rayo de luz saliente por el punto y por el centro de proyeccion. Dado este modelo, la calibración sucede de forma similar a la de una cámara, pero en lugar de tomar imágenes de un tablero de damas fijo, se proyecta un tablero en una ubicacion conocida y se toman imagenes del mismo utilizando la cámara para analizar las distorsiones. Este enfoque resulta ser una extension directa del metodo de Zhang para calibración de cámaras, por lo que toda la teoria y software disponible es reutilizado.  

Existen varias implementaciones de esta tecnica de calibracion basado en el modelo inverso de la cámara, la mayoria de ellos desarrollados por investigadores para uso propio. Puede observarse una de estas implemtaciones, BYO3D\cite{BYO3D}, en la que se proveen dos alternativeas, un \emph{Toolbox} de MATLAB\cite{MATLAB} y una extension a la biblioteca de visión computacional\footnote{computer vision} OpenCV\cite{OpenCV}.

\paragraph{Análisis de deformaciones}

\subsubsection{Kinect}

\emph{Kinect} es un dispositivo que se puede adicionar a la consola \emph{Xbox360} de \emph{Microsoft}. El objetivo de este dispositivo es permitir que el usuario interactúe con la consola utilizando solo el movimiento de su cuerpo. Para lograr esto, \emph{Kinect} aplica distintas técnicas de procesamiento de imágenes, considerando ubicaciones, posturas y distancias. El \emph{hardware} de \emph{Kinect} no consiste tan solo de una cámara, sino que tiene adicionalmente un emisor de infrarrojos que en base a la deformación del haz de luz determina la distancia de cada punto de la imagen capturada. Posteriormente, combina la información visual para tener una noción bastante precisa de los movimientos del usuario. A mediados del 2011 fue presentada una interfaz de programación gratuita que permite utilizar \emph{Kinect} de forma directa en aplicaciones no licenciadas para diferentes propósitos, no solo el de los videojuegos.

\begin{figure}[H]
  \centering
    \includegraphics[width=0.7\textwidth]{./Cap2_videomapping/kinect.PNG}
  \caption{Componentes de un sensor Kinect}%ref%
  \label{fig:Kinect}
\end{figure}

Para el análisis de las deformaciones de los rayos y construir el mapa de profundidad, \emph{Kinect} utiliza la previamente mencionada técnica de luz estructurada. Mediante el emisor infrarrojo con el que viene equipado, se proyectan los patrones de luz por toda la escena y utilizando la cámara de profundidad se analizan las distorsiones.
También se cuenta con una cámara convencional para analizar objetos o personas en la escena y la detección de colores.
En cuanto a los usos específicos orientados a la reconstrucción tridimensional de objetos, el proyecto \emph{KinectFusion} \cite{KinectFusion}, actualmente patrocinado por \emph{Microsoft}, logra muy buenos resultados.%extender kinectfusion%

\subsection{Procesamiento de nube de puntos}

Al utilizar técnicas de obtención de geometría lo que se obtiene son nubes de puntos. Es necesario simplificar este modelo con el objetivo de eliminar redundancia y aumentar la velocidad en el procesamiento de los datos. Un problema adicional al volumen de la información obtenida es que comúnmente se introduce ruido. Para solucionar este problema se realiza un suavizado en el procesamiento de la nube de puntos \cite{PCloudSimplify}.

Las heurísticas utilizadas se pueden clasificar como \cite{PntCloud}:
\begin{itemize}
   \item \emph{Clustering methods}: consiste en obtener grupos de la nube de puntos en donde cada grupo se remplaza por un conjunto de puntos representativos en él. Los grupos se pueden construir utilizando un enfoque incremental en el cual estos son creados iniciando por un punto aleatorio y agregando puntos vecinos hasta llegar a una cantidad establecida de elementos o un enfoque jerárquico en donde se particiona el conjunto de puntos recursivamente hasta conseguir grupos de un tamaño predefinido.
   \item \emph{Iterative simplification}: se recorren iterativamente los puntos de la nube contrayendo parejas en un único punto. Se evalúa el error introducido, utilizando mínimos cuadrados, que se genera en la contracción comparándolo con el error que se obtendría al contraerse con otro punto vecino eligiendo la contracción que introduce menor error al sistema. La simplificación se da por finalizada por haber logrado la cantidad de puntos deseada o por superar una cota de error a introducir en el sistema.
   \item \emph{Particle simulation}: se generan nuevos puntos que sustituyen la nube de puntos original. Se generan conjuntos de partículas que se mueven aleatoriamente en la superficie hasta lograr un balance. %Luego, utilizando el algoritmo de point–repulsion se definen en las zonas que hay mayor colisiones.%
\end{itemize}

Luego de simplificar la nube de puntos se construye un modelo tridimensional a partir de ella utilizando mallas de triángulos. La razón principal es la simplicidad de los algoritmos que dibujan triángulos. Esto permite que sean implementados fácilmente en hardware además del beneficio de que cualquier polígono con más de tres caras puede representarse como un conjunto de triángulos \cite{PCloudTriangle}.

\subsection{Tratamiento de malla}

Los dispositivos de captura de información tridimensional estudiados entregan la información en forma de nube de puntos. Es por ello que previo a la manipulación de la información tridimensional, es necesario procesar dicha nube de puntos para convertirla a representaciones más manejables como por ejemplo mallas triangulares.
Un algoritmo de procesamiento de malla se basa en tomar como entrada una nube de puntos, realizar un sub-muestreo y suavizado de la misma, calcular las normales en cada punto de la nube y finalmente, aplicar algoritmos de reconstrucción de malla. Esto ha sido estudiado e implementado en las bibliotecas \emph{VcgLib} \cite{VCGLib} y \emph{CGAL} \cite{CGAL}.

\begin{figure}[H]
  \centering
    \includegraphics[width=0.5\textwidth]{./Cap2_videomapping/malla-flow.png}
  \caption{Típico flujo para el procesamiento de nubes de puntos. Fuente: \emph{CGAL}}%ref%
  \label{fig:Mesh-CGAL}
\end{figure}

Para la implementación de este módulo se utilizan algoritmos incluidos en \emph{VcgLib}. Para visualizar y evaluar los resultados esperados fue utilizada la aplicación de código abierto para la manipulación de mallas tridimensionales en diferentes formatos \emph{MeshLab} \cite{MeshLab}. Particularmente se utilizan los algoritmos de muestreo \emph{Poisson-disk} para reducir y normalizar los puntos de la malla inicial, \emph{normal extrapolation} para el cálculo de normales y reconstrucción de superficies de Poisson para la reconstrucción de la malla final.

\subsubsection{Muestreo \emph{Poisson-disk}}

%falta referencia%
El muestreo de variables aleatorias es una técnica utilizada para una gran variedad de aplicaciones como procesamiento de imágenes y geometrías. Particularmente, el muestreo \emph{Poisson-disk} se utiliza para la ubicación aleatoria de objetos en mundos artificiales, algoritmos de texturas procedurales y procesamiento de geometrías o mallas.%REFERENCIAR%
Esta técnica genera conjuntos de puntos con la propiedad de obtener puntos suficientemente juntos pero con la restricción de no estar más próximos unos de otros que una distancia mínima $R$ predeterminada.

\begin{figure}[H]
  \centering
    \includegraphics[width=0.5\textwidth]{./Cap2_videomapping/malla-poisson.png}
  \caption{a) Posición $x$ e $y$ generadas aleatoriamente. b) Imagen dividida en celdas. Puntos aleatorios generados en cada celda. c) Muestreo \emph{Poisson-disk} en dos dimensiones.}
  \label{fig:Mesh-Poisson}
\end{figure}

En líneas generales, este algoritmo genera puntos alrededor de los ya existentes en la muestra y valida si pueden ser agregados al conjunto final en caso de no violar la regla de la mínima distancia a los vecinos. Se genera una grilla en dos o tres dimensiones, dependiendo del escenario de aplicación, en la cuál cada celda contendrá al final del proceso a lo sumo un punto. Una grilla adicional es utilizada para realizar búsquedas rápidas y dos conjuntos de puntos son mantenidos durante el procesamiento para poder diferenciar los que han sido generados y los que aún necesitan procesamiento.
La implementación realizada en \emph{VcgLib} recibe tres parámetros:
\begin{itemize}
	\item 1) La cantidad de puntos en la muestra. En este caso el radio de cercanía es calculado en base a este parámetro.
	\item 2) El radio, que es a su vez utilizado para calcular el tamaño de la muestra óptimo en base a la malla inicial.
	\item 3) Sub muestreo: indica si la muestra de Poisson es un subconjunto de la muestra inicial o si se deberán generar nuevos puntos aleatoriamente.
\end{itemize}

\begin{figure}[H]
  \centering
    \includegraphics[width=0.5\textwidth]{./Cap2_videomapping/malla-poissongui.png}
  \caption{Configuración de parámetros del algoritmo \emph{Poisson-disk}}
  \label{fig:Mesh-PoissonGui}
\end{figure}

\subsubsection{Reconstrucción de normales}

Este algoritmo computa las normales en cada elemento de un conjunto de puntos sin la necesidad de explorar la conectividad de los triángulos. Por ello es muy útil para objetos tridimensionales sin información de caras.
Se detalla un pseudo-código del método:

%Figura: planos tangentes%

\paragraph{Paso 1: identificar los planos tangentes para aproximar localmente la superficie y estimar así los vectores normales.}
Para cada vértice:
	\begin{itemize}
		\item Calcular el centro geométrico del plano tangente en el punto como el promedio de los $K$ puntos más cercanos.
		\item Calcular la normal asociada al centro geométrico. Se utiliza la matriz de covarianza en el punto, contemplando los mismos $K$ vecinos más cercanos de la muestra y los valores y vectores propios de la matriz de covarianza. Finalmente, ordenando los vectores propios, la estimación del vector perpendicular corresponde al vector propio de menor valor. Este método es conocido como \emph{Principal Component Analysis (PCA)}\footnote{PCA}.
	\end{itemize}

\paragraph{Paso 2: construir un grafo en donde cada punto está conectado a los $K$ vecinos más cercanos (grafo de Riemannian)}
Se crea un grafo en cuyos nodos se guardan todas las aristas incidentes a los $K$ vecinos más cercanos. A cada arista se le asigna un peso igual al valor absoluto del producto escalar de la normal en el punto con la normal en cada uno de los $K$ vecinos:
   $$fabs(nodoActual->normal . K\_Vecinos[n]->normal)$$
\paragraph{Paso 3: calcular el árbol de expansión mínimo sobre el grafo de Riemannian y recorrerlo para orientar las normales.}
Dado un grafo conexo, no dirigido, con sus aristas con un peso asignado, se llama árbol de expansión mínimo al sub-grafo con forma de árbol que conecta todos los nodos con un peso total mínimo conteniendo todos los nodos del grafo inicial. El grafo de entrada es el construido en el paso anterior y se utiliza el algoritmo de Kruskal\footnote{Kruskal}, uno de los varios algoritmos que resuelven el problema de encontrar un árbol de expansión mínima de un grafo.
Una vez construido el árbol de expansión, lo único que se hace es recorrerlo en orden e invertir el sentido de los vectores normales en caso de ser necesario. La condición para efectuar dicha corrección se basa en el ángulo del nodo siendo inspeccionado en comparación a todas las direcciones de las normales de los vecinos conectados a dicho nodo.

\begin{figure}[H]
  \centering
    \includegraphics[width=0.5\textwidth]{./Cap2_videomapping/malla-normalextrapolation.png}
  \caption{Configuración de parámetros para reconstrucción de normales}
  \label{fig:Mesh-Extrapolation}
\end{figure}

\subsubsection{Reconstrucción de malla de Poisson}

Finalmente, para reconstruir la malla a partir de la nube de puntos y sus normales, se utiliza el algoritmo de reconstrucción de Poisson.
Se computa una función indicadora $\chi$ definida de la siguiente forma:
%Se computa una función indicadora $\chi$ en tres dimensiones definida de la siguiente forma:%

$$
\left\{ \begin{array}{rl}
 \chi = 1 & \mbox{ si puntos dentro del modelo} \\
 \chi = 0 & \mbox{ si puntos fuera del modelo}
       \end{array} \right.
$$

Luego, se obtiene una reconstrucción de la superficie mediante la extracción de la superficie de nivel en 3 dimensiones\footnote{ISO-surface} al nivel apropiado.
La estrategia se basa en la estrecha relación que hay entre los puntos de la muestra, orientados por sus normales, y la función indicadora de la muestra. Específicamente el gradiente de la función indicadora es un espacio de vectores, de valor nulo en todo el espacio excepto en puntos cercanos a la superficie, donde son iguales al vector normal a ella.
Es por eso que puntos orientados pueden ser vistos como muestras del gradiente de la función indicadora del modelo tridimensional en cuestión y es por este mismo motivo que el problema de reconstrucción de una malla puede ser visto como un problema de Poisson estándar, es decir, computar la función escalar $F$ cuya divergencia del gradiente\footnote{Laplaciano} se iguala a la divergencia del espacio de vectores de las normales.

\begin{figure}[H]
  \centering
    \includegraphics[width=0.5\textwidth]{./Cap2_videomapping/malla-poissonreconstruction.png}
  \caption{Configuración de parámetros para reconstruccion de malla de Poisson}
  \label{fig:Mesh-Normals}
\end{figure}

\subsubsection{Pruebas y resultados}

%Esta sección va dentro de lo que implementamos, recien aca salta que lo hicimos nosotros%

Para validar el correcto funcionamiento de esta técnica mediante los tres algoritmos descritos, fue utilizada una malla inicial de 5021 vértices y 9608 caras triangulares. Cabe destacar que si bien se ha mencionado que la malla de entrada debe ser simplemente una nube de puntos, se pueden utilizar mallas con caras, solo que estas serán ignoradas e incluso eliminadas de la malla de salida del primer paso del procesamiento (muestreo de \emph{Poisson-disk}).
Luego de experimentar con varios juegos de datos iniciales durante varias ejecuciones del procesamiento, se fijaron de manera personalizada para la malla de entrada algunos parámetros clave. Dado que la muestra inicial tiene alrededor de cinco mil puntos, fueron elegidas cinco mil muestras para el algoritmo de \emph{Poisson-disk}. Luego, para la extrapolación de normales se utilizan $K=15$ vecinos para la toma de decisiones locales de aproximación.
La aplicación de estos algoritmos resultó ser lo esperado en términos estructurales de cada malla procesada en cada uno de los pasos.
%Puede ir a trabajo futuro: No se llegó a procesar mallas de ambientes tridimensionales escaneados para luego ser mapeados con la herramienta.%

\begin{table}
\begin{center}
\begin{tabular}{|l||cc|} \hline
	Fase & Vértices & Caras \\
	Nube inicial & 5021 & 9608 \\
	Poisson-disk & 1776 & 0 \\
	Extrapolación Normales & 1776 & 0 \\
	Reconstrucción de Poisson & 1959 & 3910 \\ \hline %hay 1959 puntos en lugar de 1776, por que?%
\end{tabular}
\caption{Comparación de estructura de mallas de entrada y salida en cada fase}
\end{center}
\end{table}

\begin{figure}[H]
  \centering
    \includegraphics[width=0.8\textwidth]{./Cap2_videomapping/malla-nubepuntos.png}
  \caption{1) Nube de puntos inicial con 5021 vértices. 2) Resultado de muestreo \emph{Poisson-disk} con 1776 vértices. 3) Luego de extrapolar normales y reconstruir la malla con 1959 vértices y 3910 caras.}
  \label{fig:Mesh-Results}
\end{figure}
 
%Se observaron buenos tiempos computacionales de respuesta. Si bien la malla utilizada no es de un tamaño considerable, estamos hablando de algoritmos de orden relativamente alto.%
